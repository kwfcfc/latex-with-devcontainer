\documentclass[UTF8]{article}
% we don't use the default fandol font, so we supress it here.
\usepackage[fontset=none]{ctex}
\usepackage[a4paper, margin=2.5cm]{geometry}
\usepackage{fontspec}
\setmainfont{TeX Gyre Pagella}
\setCJKmainfont{Noto Sans CJK SC}
\setCJKmonofont{Noto Sans Mono CJK SC}

\author{kwfcfc}
\date{}
\title{容器里的\LaTeX }
\begin{document}
\maketitle

你好,世界!

Hello, world!

A quick brown fox jumps over a lazy dog.

\section{\LaTeX in devcontainer}

使用DevContainer搭建\LaTeX 集成开发环境。安装了一些自定义字体,使用xelatex搭建中文环境。此
外安装了几个开源的英文和中文字体。由于容器里没有Fandol等默认的黑体和宋体字体,在使用CTeX包时将
字体集设计为空。

\section{TeXLive as package manager}

使用TexLive的包管理器,因此用户可以使用tlmgr在终端安装或者删除包。需要执行:
\texttt{sudo tlmgr path add}
将安装的包加入\texttt{\$PATH}。

此外,优先使用xelatexmk,还保留了用biber的工作流,以及用latexmk的工作流。

\end{document}